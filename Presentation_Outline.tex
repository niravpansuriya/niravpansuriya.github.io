 % ===========================================================================
% Title:
% ---------------------------------------------------------------------------
% to create Type I fonts type "dvips -P cmz -t letter <filename>"
% ===========================================================================
\documentclass[11pt]{article}       %--- LATEX 2e base
\usepackage{latexsym}               %--- LATEX 2e base
%---------------- Wide format -----------------------------------------------
\textwidth=6in \textheight=9in \oddsidemargin=0.25in
\evensidemargin=0.25in \topmargin=-0.5in
%--------------- Slide --------------------------------------------------
\newenvironment{slide}[1]        {\section{#1} \begin{itemize}}%
                                 {\end{itemize}}
                                 % usage: \begin{slide}{---SLIDE TITLE---}
                                 %        \item ...
                                 %        \item ...
                                 %        \item ...
                                 %        \end{slide}
% ===========================================================================
\begin{document}
% ===========================================================================

% ############################################################################
% Title
% ############################################################################
\newcommand{\SubItem}[1]{
    {\setlength\itemindent{15pt} \item[-] #1}
}
\title{PRESENTATION OUTLINE: A Parallel Jacobi-Embedded Gauss-Seidel Method}


% ############################################################################
% Author(s) (no blank lines !)
\author{
% ############################################################################
Nirav Pansuriya\\
School of Computer Science\\
Carleton University\\
Ottawa, Canada K1S 5B6\\
{\em niravchhaganbhaipan@cmail.carleton.ca}
% ############################################################################
} % end-authors
% ############################################################################

\maketitle

% ############################################################################
\begin{slide}{Introduction}
\item Student Name
\item Student Id Number
\item Title of the project
\end{slide}


% ############################################################################
\begin{slide}{Computer Methods to solve linear equations}
\item Direct Method
\item Iterative Method
\item Name of methods which are used in my work
\end{slide}

\begin{slide}{Jacobian Method}
\item Quick explaination of how this method workds?
\item Advantages and Disadvantages (as they are used in my work)
\end{slide}

\begin{slide}{Gauss Seidel Method}
\item Quick explaination of how this method workds?
\item Advantages and Disadvantages
\end{slide}

\begin{slide}{Row based parallel Gauss Seidel Method}
\item Quick explaination of how this method workds?
\item Different ways to implement this method in parallel system
\item Comparision between those ways
\end{slide}

\begin{slide}{PJG Method}
\item Quick explaination of how this method workds?
\item What is block size P in this method?
\end{slide}

\begin{slide}{Block Size P}
\item Significance of block size P
\item Analysis of PJG method with different block size P
\item How I have used this analysis in my work?
\end{slide}

\begin{slide}{Questions for audience}
\item Three question for audience to make sure they understand research talk till now
\end{slide}

\begin{slide}{My Approch}
\item My hypothesis of this work
\item Results of the hypothesis
\end{slide}

\begin{slide}{Environment Specifications}
\item Specifications of GPU
\end{slide}

\begin{slide}{Results}
\item Comparision of all mentiond methods with my approch in terms of performance
\end{slide}

\begin{slide}{Conclusion}
\item Conclusion of my work that audience should remember from research talk
\end{slide}

\begin{slide}{Questions for audience}
\item Two questions for audience to make sure that they understand my research talk
\end{slide}
% ############################################################################
% Bibliography
% ############################################################################
\bibliographystyle{plain}     %loads my-bibliography.bib

% ============================================================================
\end{document}
% ============================================================================
